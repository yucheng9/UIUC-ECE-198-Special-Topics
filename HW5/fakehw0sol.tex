% ---------
%  Compile with "pdflatex hw0".
% --------
%!TEX TS-program = pdflatex
%!TEX encoding = UTF-8 Unicode

\documentclass[11pt]{article}
\usepackage{jeffe,handout,graphicx}
\usepackage[utf8]{inputenc}		% Allow some non-ASCII Unicode in source

% =========================================================
%   Define common stuff for solution headers
% =========================================================
\pagenumbering{arabic}
\Class{ECE 198}
\Semester{Fall 2019}
\Authors{1}
\AuthorOne{Jin Yucheng}{yucheng9}
%\AuthorTwo{Friday Caliban}{fcaliban}
%\AuthorThree{Duncan Quagmire}{dquagmir}
%\Section{}
\usepackage{float}

% =========================================================
\begin{document}

% ---------------------------------------------------------


\HomeworkHeader{5}{1-2}	% homework number, problem number

\begin{solution}
%These are, without exception, inappropriate inquiries, a phrase which here means “all the wrong questions”.  Here are the questions you should have asked instead:
%\begin{enumerate}[(a)]
%\item Why would someone say something was stolen when it was never theirs to begin with?
%\item How could someone who was missing be in two places at once?
%\item Why would someone destroy one building when they really wanted to destroy another?
%\end{enumerate}
%\begin{enumerate} [(a)]

\item (1) Choose an undecidable problem that is not stated in terms of Turing Machines and provide a one-paragraph, less than half-page sketch of its undecidability proof.
 \item
\noindent\fbox{
    \parbox{\textwidth}{
\item
Reference: "How is proving a context free language to be ambiguous undecidable?" \textit{StackExchange}. 2011. <https://cstheory.stackexchange.com/questions/4352/how-is-proving-a-context-free-language-to-be-ambiguous-undecidable>
\item The undecidable problem I want to sketch the proof is, it is undecidable whether a context-free grammar is ambiguous.
\item The general idea is that we could reduce it from Post's Correspondence Problem (PCP). Suppose we can decide the language $L(G)$ generated by CFG $G$ is ambiguous, we somehow construct a CFG and if the language is ambiguous then there is a derivation of some string in two different ways, but this leads to the result that we have a solution to PCP; similarly, if there is no ambiguity, then PCP cannot be solved. Therefore, we've reduced the original problem to PCP, and since PCP is undecidable, we can prove that it is undecidable whether a context-free grammar is ambiguous.
    }
}
\item
\item (2) Prove by induction that in a rooted tree, every node is reached by a DFS starting from the root.
 \item
\noindent\fbox{
    \parbox{\textwidth}{
\item
A tree is a \textit{\textbf{connected}} undirected graph with no simple circuits, and a rooted tree is a tree in which one vertex has been designated as the root and \textit{\textbf{every edge is directed away from the root}}. A preorder traversal is a DFS that visits all tree nodes, given a rooted tree $T$ and its root $r$ (this rooted tree is not empty), preorder traversal works as follows:
\item \textbf{def} \underline{PreorderTraversal$(T, r)$}:
\item \qquad $\text{  } 1:$ visit $r$
\item \qquad $\text{  } 2:$ \textbf{for} each child $c$ of $r$ from left to right:
\item \qquad $\text{  } 3:$ \qquad $T_c$ = subtree with $c$ as its root 
\item \qquad $\text{  } 4:$ \qquad PreorderTraversal$(T_c, c)$
\item \textbf{\textit{Proof of Correctness}}
\begin{itemize}
\item Basis Step: when the rooted tree $T$ only has one node, $r$, then preorder traversal visits $r$.
\item Inductive Step: Suppose preorder traversal visits all nodes of a rooted tree $T'$, if $T'$ is a subtree of another rooted tree $T$. Since $T$ can be expressed as a root $r$ with $n$ children, each child $i$ forms a subtree $T_{i}$. For any subtree, $T_{1}$, $T_{2}$, ..., $T_{n}$, preorder traversal visits all nodes, therefore, PreorderTraversal$(T, r)$ first visits $r$ and then calls PreorderTraversal$(T_{i}, r_{i})$ for every subtree $i$, and PreorderTraversal$(T, r)$ visits $r$ and every node in its subtrees.
\end{itemize}
So every node is reached by a DFS starting from the root in a rooted tree.
  }
}
\end{solution}

\end{document}